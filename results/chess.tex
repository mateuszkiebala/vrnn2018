\documentclass[10pt,twocolumn,letterpaper]{article}

\usepackage{cvpr}
\usepackage{times}
\usepackage{epsfig}
\usepackage{graphicx}
\usepackage{amsmath}
\usepackage{amssymb}

% Include other packages here, before hyperref.

% If you comment hyperref and then uncomment it, you should delete egpaper.aux
% before re-running latex.  (Or just hit 'q' on the first latex run, let it
% finish, and you should be clear).
\usepackage[breaklinks=true,bookmarks=false]{hyperref}

\cvprfinalcopy % *** Uncomment this line for the final submission

\def\cvprPaperID{****} % *** Enter the CVPR Paper ID here
\def\httilde{\mbox{\tt\raisebox{-.5ex}{\symbol{126}}}}

% Pages are numbered in submission mode, and unnumbered in camera-ready
% \ifcvprfinal\pagestyle{empty}\fi
\setcounter{page}{4321}
\begin{document}

%%%%%%%%% TITLE
\title{VRNN 2018 - Predicting legal chess moves}

\author{Mateusz Kiebala\\
University of Warsaw\\
{\tt\small mk359758@students.mimuw.edu.pl}
% For a paper whose authors are all at the same institution, omit the following
% lines up until the closing ``}''. Additional authors and addresses can be
% added with ``\and'', just like the second author. To save space, use either
% the email address or home page, not both
\and
Tomasz Knopik\\
University of Warsaw\\
{\tt\small tk359778@students.mimuw.edu.pl}
\and
Janusz Marcinkiewicz\\
University of Warsaw\\
{\tt\small jm360338@students.mimuw.edu.pl}}

\maketitle
%\thispagestyle{empty}

%%%%%%%%% ABSTRACT
\begin{abstract}
   One of the recent problems that gained a lot of attention in Deep Learning
   field was playing Go and Chess at the professional level. The models for these
   problems were assigning the probability of the win for certain moves and
   positions where the moves were generated deterministically. In our work we are
   checking if it is possible to learn a CNN network the rules of chess
   providing only image of the board before and after the move. We test two
   different models architectures: single CNN model which processes two input
   images as one concatenated; and Siamese CNN model which takes two images
   separately, extracts features combines them together predicting the result.
   These two approaches achieve 90\% on training set and 85\% on validation set.
\end{abstract}

%%%%%%%%% BODY TEXT
\section{Introduction}

Recent achievements in Deep Learning allowed researchers to beat top players in
Go and Chess using Deep Learning models. To learn model they have used
techniques like Reinforcement Learning and Deep Convolutional Neural Networks to
predict win percentages for given positions.

We believe that this problem is super important. If model can learn the rules of
the chess, maybe it also can learn the rules of the world. Long standing
question about single formula describing the world could be solved by CNN. It is
still long way ahead of us but we believe we can make it and our work is first
step towards this achievement.

\section{Related work}

We have struggled to find any papers or work done in this area. All the papers
are targeting learning model to evaluate positions but not how to play
the game. In end-to-end chess playing paper~\cite{DeepChess} authors discuss learning
model to not only evaluate positions but also determining if the move is correct.
We have used key insights from the paper to improve our models architectures.
Another paper about predicting moves~\cite{Oshri2015PredictingMI} provided a lot
of useful information about using and not using different types of layers in our
models.

\section{Data}

Chess is deterministic game and the moves as well as positions can be easily
generated and encoded.

Data used in our experiments was generated using python chess library which
allowed to save given position as images which then we converted into numpy
arrays. We have created framework for generating random data with different
outputs which were necessary for different types of models we have tested. The
data was generated from random plays as well as positions and moves from book
games. We have combined these two sources into datasets on which our models were
tested. This way we wanted our model to prevent overfitting on specific types of
plays. Illegal moves were created randomly but putting pieces in empty spaces on
the chess board. The ratio of legal and illegal moves was around 50-50.

To train the models we have generated around 15 datasets with different random
-- book games ratio each of them containing about 5000 moves. In total we gained
about 70000 moves to train on.

\section{Methods}

In our approach, we tried two different type of CNN models.

\subsection{Single CNN model}

One of the first models we train is CNN model which consists of couple convolution
layers followed by the fully connected layers. The model, as an input, was feeded
with concatenated images: before and after move. In this model we wanted to check
if convolution layers can extract more features having more information.

\subsection{Siamese CNN model}

Siamese CNN model consisted of two identical CNN networks which output was then
concatenated and passed to the fully connected layers. This model was feeded
with two inputs: image before and after move. Our understanding was that the
model should independently extract features of each of the image and in the
output have some kind of encoding which then was feeded to fully connected
layers.

\subsection{Deterministic encoding}

In our last method we tried to see what accuracy can we achieve if we pass
encoding of the chess board to fully connected layer without need of extracting
features by convolution layers. This experiment was to see what is the bottleneck:
too shallow representation of features extracted by convolution layers or fully
connected layers. A term deterministic encoding means a chess board, encoded as
\{0,1\} vector of length 768 (there are 12 different pieces which each of 
64 positions can by occupied by).

\section{Experiments}

In the experiments we have tried different approaches for different types of
models. In both types of models as well as deterministic encoding we achieved
~85\% on validation set.

Interestingly, the deterministic encoding had similar accuracy as the other models
what suggest that convolution layers did a great job when extracting features
and the bottleneck in the accuracy were fully connected layers. Worth noting is the fact, 
that thanks to the compact encoding representation, we were able to train a network on a 
much bigger dataset of roughly 200,000 moves.
However we observed that it did not have signiticant impact on results, as final
accuracy reached 87%.
With this information we tried to implemented different set of fully 
connected layers but without further success. 

\subsection{Results by method}


\begin{center}
    \begin{tabular}{ | p{5em} | p{5em} | p{5em} | p{5em} |}
    \hline
    Method & Dataset size & No. epochs & Accuracy \\ [0.5ex] 
    \hline \hline
    Single & 15,000 & 10k & 50\% \\ 
    \hline
    Siamese & 9C & 19C & 50\% \\ \hline
    Single gray scale & 10C & 21C & 87\% \\
    \hline
    \end{tabular}
\end{center}

\section{Conclusions}

??

\begin{quote}
\begin{center}
     An analysis of the frobnicatable foo filter.
\end{center}

   In this paper we present a performance analysis of the paper of Smith \etal
   [1], and show it to be inferior to all previously known methods.  Why the
   previous paper was accepted without this analysis is beyond me.

   [1] Smith, L and Jones, C. ``The frobnicatable foo filter, a fundamental
   contribution to human knowledge''. Nature 381(12), 1-213.
\end{quote}

If you are making a submission to another conference at the same time, which
covers similar or overlapping material, you may need to refer to that submission
in order to explain the differences, just as you would if you had previously
published related work.  In such cases, include the anonymized parallel
submission~\cite{Authors14} as additional material and cite it as
\begin{quote}
[1] Authors. ``The frobnicatable foo filter'', F\&G 2014 Submission ID 324,
Supplied as additional material {\tt fg324.pdf}.
\end{quote}

Finally, you may feel you need to tell the reader that more details can be found
elsewhere, and refer them to a technical report.  For conference submissions,
the paper must stand on its own, and not {\em require} the reviewer to go to a
techreport for further details.  Thus, you may say in the body of the paper
``further details may be found in~\cite{Authors14b}''.  Then submit the
techreport as additional material. Again, you may not assume the reviewers will
read this material.

Sometimes your paper is about a problem which you tested using a tool which is
widely known to be restricted to a single institution.  For example, let's say
it's 1969, you have solved a key problem on the Apollo lander, and you believe
that the CVPR70 audience would like to hear about your solution.  The work is a
development of your celebrated 1968 paper entitled ``Zero-g frobnication: How
being the only people in the world with access to the Apollo lander source code
makes us a wow at parties'', by Zeus \etal.

You can handle this paper like any other.  Don't write ``We show how to improve
our previous work [Anonymous, 1968].  This time we tested the algorithm on a
lunar lander [name of lander removed for blind review]''. That would be silly,
and would immediately identify the authors. Instead write the following:
\begin{quotation}
\noindent
   We describe a system for zero-g frobnication.  This system is new because it
   handles the following cases: A, B.  Previous systems [Zeus et al. 1968]
   didn't handle case B properly.  Ours handles it by including a foo term in
   the bar integral.

   ...

   The proposed system was integrated with the Apollo lunar lander, and went all
   the way to the moon, don't you know.  It displayed the following behaviours
   which show how well we solved cases A and B: ...
\end{quotation}
As you can see, the above text follows standard scientific convention, reads
better than the first version, and does not explicitly name you as the authors.
A reviewer might think it likely that the new paper was written by Zeus \etal,
but cannot make any decision based on that guess. He or she would have to be
sure that no other authors could have been contracted to solve problem B.
\medskip

\noindent
FAQ\medskip\\
{\bf Q:} Are acknowledgements OK?\\
{\bf A:} No.  Leave them for the final copy.\medskip\\
{\bf Q:} How do I cite my results reported in open challenges? {\bf A:} To
conform with the double blind review policy, you can report results of other
challenge participants together with your results in your paper. For your
results, however, you should not identify yourself and should not mention your
participation in the challenge. Instead present your results referring to the
method proposed in your paper and draw conclusions based on the experimental
comparison to other results.\medskip\\



\begin{figure}[t]
\begin{center}
\fbox{\rule{0pt}{2in} \rule{0.9\linewidth}{0pt}}
   %\includegraphics[width=0.8\linewidth]{egfigure.eps}
\end{center}
   \caption{Example of caption.  It is set in Roman so that mathematics (always
   set in Roman: $B \sin A = A \sin B$) may be included without an ugly clash.}
\label{fig:long}
\label{fig:onecol}
\end{figure}

\subsection{Miscellaneous}

\noindent
Compare the following:\\
\begin{tabular}{ll}
 \verb'$conf_a$' &  $conf_a$ \\
 \verb'$\mathit{conf}_a$' & $\mathit{conf}_a$ \end{tabular}\\
See The \TeX book, p165.

The space after \eg, meaning ``for example'', should not be a sentence-ending
space. So \eg is correct, {\em e.g.} is not.  The provided \verb'\eg' macro
takes care of this.

When citing a multi-author paper, you may save space by using ``et alia'',
shortened to ``\etal'' (not ``{\em et.\ al.}'' as ``{\em et}'' is a complete
word.) However, use it only when there are three or more authors.  Thus, the
following is correct: `` Frobnication has been trendy lately. It was introduced
by Alpher~\cite{Alpher02}, and subsequently developed by Alpher and
Fotheringham-Smythe~\cite{Alpher03}, and Alpher \etal~\cite{Alpher04}.''

This is incorrect: ``... subsequently developed by Alpher \etal~\cite{Alpher03}
...'' because reference~\cite{Alpher03} has just two authors.  If you use the
\verb'\etal' macro provided, then you need not worry about double periods when
used at the end of a sentence as in Alpher \etal.

For this citation style, keep multiple citations in numerical (not
chronological) order, so prefer \cite{Alpher03,Alpher02,Authors14} to
\cite{Alpher02,Alpher03,Authors14}.


\begin{figure*}
\begin{center}
\fbox{\rule{0pt}{2in} \rule{.9\linewidth}{0pt}}
\end{center}
   \caption{Example of a short caption, which should be centered.}
\label{fig:short}
\end{figure*}

%-------------------------------------------------------------------------
\subsection{Type-style and fonts}

Wherever Times is specified, Times Roman may also be used. If neither is
available on your word processor, please use the font closest in appearance to
Times to which you have access.

MAIN TITLE. Center the title 1-3/8 inches (3.49 cm) from the top edge of the
first page. The title should be in Times 14-point, boldface type. Capitalize the
first letter of nouns, pronouns, verbs, adjectives, and adverbs; do not
capitalize articles, coordinate conjunctions, or prepositions (unless the title
begins with such a word). Leave two blank lines after the title.

AUTHOR NAME(s) and AFFILIATION(s) are to be centered beneath the title and
printed in Times 12-point, non-boldface type. This information is to be followed
by two blank lines.

The ABSTRACT and MAIN TEXT are to be in a two-column format.

MAIN TEXT. Type main text in 10-point Times, single-spaced. Do NOT use
double-spacing. All paragraphs should be indented 1 pica (approx. 1/6 inch or
0.422 cm). Make sure your text is fully justified---that is, flush left and
flush right. Please do not place any additional blank lines between paragraphs.

Figure and table captions should be 9-point Roman type as in
Figures~\ref{fig:onecol} and~\ref{fig:short}.  Short captions should be centred.

\noindent Callouts should be 9-point Helvetica, non-boldface type. Initially
capitalize only the first word of section titles and first-, second-, and
third-order headings.

FIRST-ORDER HEADINGS. (For example, {\large \bf 1. Introduction}) should be
Times 12-point boldface, initially capitalized, flush left, with one blank line
before, and one blank line after.

SECOND-ORDER HEADINGS. (For example, { \bf 1.1. Database elements}) should be
Times 11-point boldface, initially capitalized, flush left, with one blank line
before, and one after. If you require a third-order heading (we discourage it),
use 10-point Times, boldface, initially capitalized, flush left, preceded by one
blank line, followed by a period and your text on the same line.


%-------------------------------------------------------------------------
\subsection{References}

List and number all bibliographical references in 9-point Times, single-spaced,
at the end of your paper. When referenced in the text, enclose the citation
number in square brackets, for example~\cite{Authors14}.  Where appropriate,
include the name(s) of editors of referenced books.

\begin{table}
\begin{center}
\begin{tabular}{|l|c|}
\hline
Method & Frobnability \\
\hline\hline
Theirs & Frumpy \\
Yours & Frobbly \\
Ours & Makes one's heart Frob\\
\hline
\end{tabular}
\end{center}
\caption{Results.   Ours is better.}
\end{table}

%-------------------------------------------------------------------------
\subsection{Illustrations, graphs, and photographs}

All graphics should be centered.  Please ensure that any point you wish to make
is resolvable in a printed copy of the paper.  Resize fonts in figures to match
the font in the body text, and choose line widths which render effectively in
print.  Many readers (and reviewers), even of an electronic copy, will choose to
print your paper in order to read it.  You cannot insist that they do otherwise,
and therefore must not assume that they can zoom in to see tiny details on a
graphic.

When placing figures in \LaTeX, it's almost always best to use
\verb+\includegraphics+, and to specify the  figure width as a multiple of the
line width as in the example below {\small\begin{verbatim}
\usepackage[dvips]{graphicx} ...
   \includegraphics[width=0.8\linewidth]
                   {myfile.eps}
\end{verbatim}
}


{\small
\bibliographystyle{ieee}
\bibliography{egbib}
}

\end{document}
